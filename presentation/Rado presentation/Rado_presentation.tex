\documentclass{beamer}
\setbeamercolor{block title}{use=structure,fg=white,bg=structure.fg!75!black}
\setbeamercolor{block body}{parent=normal text,use=block title,bg=block title.bg!10!bg}

\usetheme{default}

\title{Rado's Theorem}
\date{\today}
\author{Jack, Chang}
\usepackage{amsmath}
\usepackage{amsfonts} % Fonts like \mathbb and \mathcal
\usepackage{amssymb}
\usepackage{hyperref} % for hyperlinks
\usepackage{amsthm}
\usepackage{diagbox}
\newtheorem{conjecture}[theorem]{Conjecture}
\newcommand{\R}{\mathbb R} %REALS
\newcommand{\C}{\mathbb C} %COMPLEX
\newcommand{\N}{\mathbb N} %NATURAL NUMBERS
\newcommand{\Q}{\mathbb Q} %RATIONALS
\newcommand{\Z}{\mathbb Z} %INTEGERS
\newcommand{\V}{\mathbb V} 



\begin{document}

\begin{frame}[plain]
    \maketitle
\end{frame}

% Brief introduction
\begin{frame}{Introduction}
    \begin{itemize}
        \item Schur's theorem and Schur numbers. 
        \pause
        \item Schur's theorem was taken further by his Ph.D student Richard Rado.
        \pause
        \item In 1930, Rado determined that linear equations in the form of $\sum^k_{i=1}c_ix_i=0$ are guaranteed to have monochromatic solutions under any finite coloring of $\Z^+$.
    \end{itemize}
\end{frame}


% Preliminaries
\begin{frame}{Definitions}
    \begin{definition}[Coloring]
        An $r-$coloring of a set $S$ is a function $\chi : S \to C$, where $|C| = r$.
    \end{definition}
    \pause
    \begin{definition}[Monochromatic]
        A coloring $\chi$ is monochromatic on a set $S$ if $\chi$ is constant on $S$.
    \end{definition}
    
    \pause
    \begin{example}
    	Let $\chi : [1,6] \to \{0,1\}$ be defined as $\chi(a)=0$ if $a$ is odd and $\chi(a)=1$ if $a$ is even. Then we have a 2-coloring of $[1,6]$ which is monochromatic on the sets $\{1,3,5\}$ and  $\{2,4,6\}$.
    \end{example}
    
    % showing example of coloring and monochromatic sets
\end{frame}

\begin{frame}{Definitions cont'd}
    \begin{definition}[Regularity]
        For $r\geq 1$, a linear equation $D$ is called $r-$regular if there exists $n=n(D;r)$ such that for every $r-$coloring of $[1,n]$ there is a monochromatic solution to $D$. The equation $D$ is \textit{regular} if it is $r-$regular for all $r \geq 1$.
    \end{definition}
    \pause
    \begin{definition}[Rado Number]
        %The Rado Number of an equation $D$ is a theoretical quantity associated to $D$.
        For any equation $D$, the Rado Number $R_r(D)$ is the smallest $N$ such that any $r$ coloring $\chi : \{1,2,\dots,N\} \to \{1,2,\dots, r\}$ must induce a monochromatic solution to $D$. 
    \end{definition}
\end{frame}



% Rado's theorems
\begin{frame}{Rado's Theorem}
    \begin{Theorem}
            Let $k \geq 3$ and $c_i \in \Z^+-\{0\}$ for $i = 1,2,\dots, k$. Let $D$ represent the equation $\sum^k_{i=1}c_ix_i=0$. If there exist some $i,j \in \{1,2,\dots,k\}$ such that $c_i < 0$ and $c_j > 0$, then $D$ is 2-regular. 
    \end{Theorem}
\end{frame}

% Proof
\begin{frame}[t]{Proof}
    \begin{block}{Proof}
     $D \to \sum^m_{i=1}\alpha_iy_i = \sum^n_{i=1}\beta_iz_i$, where $m \geq 2, n \geq 1,$. \\
    \pause
    We consider a subset of the solution to $D$, where $y=y_1=y_2=\cdots=y_{m-1}, w=y_m$, and $z=z_1=z_2=\cdots=z_n$. \\
    $a=\sum^{m-1}_{i=1}\alpha_i, b=a_m, c=\sum^{n}_{i=1}\beta_i$. \\
    \pause
    The equation $D$ now can be rewritten as $ay+bw=cz$.\\
    \pause
    For each of the solutions $(y,w,z)$, determine the max$(y,w,z)$.\\
    \pause
    Let $(\bar{y}, \bar{w}, \bar{z})=$min$_{\forall (y,w,z)}$ max$(y,w,z)$. \\
    \pause
    $A=$ max$(\bar{y}, \bar{w}, \bar{z})$. \\
    Note that $[1,\dots, A]$ contains a solution to $D$.
    
    \end{block}
\end{frame}

% proof cont'd
\begin{frame}[t]{Proof Cont'd}
    \begin{block}{Proof}
        There exists a 2-coloring of $\Z^+$ with no monochromatic solution to $D$. \\
        \pause
        Let $l = $lcm$(\frac{a}{gcd(a,b)}, \frac{c}{gcd(b,c)})$ so that $\frac{bl}{a}, \frac{bl}{c} \in \Z^+$. \\
        \pause
        Let $s$ be the smallest element in $\{i\cdot l:i=2,\dots, A\}$ that is blue. \\
        $s$ must exist since $\{i\cdot n:i=1,2,\dots,A\}$ is \textit{not} monochromatic for any $n\in \Z^+$. \\
        \pause
        Now, let $p\in \Z^+$, we have $t=\frac{b}{a}(s-l)p$ is blue. \\
        \pause
        Now, $q=\frac{b}{c}((s-l)p+s)$ must be red, otherwise $(t,s,q)$ would be a solution to $D$ and it is monochromatic(blue).
    \end{block}
\end{frame}

% proof cont'd
\begin{frame}[t]{Proof Cont'd}
    \begin{block}{Proof}
        To see this, let's plug $(t,s,q)$ into $D$.
        $$ay+bw=cz$$
        $$a(\frac{b}{a}(s-l)p) + bs = c(\frac{b}{c}((s-l)p+s))$$
        $$bsp - blp + bs = bsp - blp + bs$$
        \pause
        Since we know that both $l,q$ are red, $\frac{b}{a}(s-l)(p+1)$ must be blue, for otherwise $(\frac{b}{a}(s-l)(p+1), l,q)$ is another monochromatic(red) solution.
        \pause
        $$a(\frac{b}{a}(s-l)(p+1)) + bl = c(\frac{b}{c}((s-l)p+s))$$
        $$bsp - blp + bs - bl + bl = bsp - blp + bs$$
    \end{block}
\end{frame}

\begin{frame}{Proof Cont'd}
    \begin{proof}
        From before, we know that $t=\frac{b}{a}(s-l)p$ and $\frac{b}{a}(s-l)(p+1)$ are both blue. \\
		\pause        
        Since $p$ was chosen randomly in $\Z^+$, we can see that $\{i\cdot \frac{b}{a}(s-l):i=p,p+1,\dots\}$ is monochromatic. In particular, we arrive at the following: 
        $$\{i\cdot \frac{b}{a}(s-l):i=1,2,\dots,A\}$$
        is monochromatic, which contradicts our assumption. 
    \end{proof}
\end{frame}

% Other theorems
\begin{frame}{Rado's Theorems}
    \begin{itemize}
        \item It is known that $x+2y-4z=0$ is not $3-$regular. 
        \pause
    \end{itemize}
    \begin{Theorem}[Rado's Single Equation Theorem]
        Let $k\geq 2$, Let $c_i\in \Z \setminus \{0\}$, for all $i \in \{1,2,\dots, k\}$, be constants. Then 
        $$\sum^k_{i=1}c_ix_i=0$$
        is regular if and only if there exists a nonempty subset $C_s \subset \{c_i:1 \leq i \leq k\}$ such that $\sum_{d\in C_s}d=0$.  
    \end{Theorem}
\end{frame}

\begin{frame}{Rado's Theorem}
    \begin{itemize}
        \item In Rado's original paper, he conjectured that for all $r\in \Z^+$, there must exist equations that are $r-$regular but not $(r+1)-$regular. 
        \pause
        \item This conjecture has been resolved with a recent paper with the following theorem:
    \end{itemize}
    \begin{Theorem}[Alexeev, Tsimerman '10]
        For every $r\in \Z^+$, the equation
        $$\sum^r_{i=1}\frac{2^i}{2^i-1}x_i = (\sum^r_{i=1}\frac{2^i}{2^i-1}-1)x_r+1$$
        is $r-$regular but not $(r+1)-$regular. 
    \end{Theorem}
\end{frame}

% Another Rado's theorem
\begin{frame}[t]{Another View to Rado's Theorem}
    \begin{definition}
        Let $A$ be an $m\times n$ matrix with rational entries. $A$ is partition regular(PR) if whenever $\N$ is finitely colored, there exists a monochromatic solution $x\in \N^n$ where $Ax=0$.
    \end{definition}
    \pause
    \begin{example}
        $A = [1, 1, -1]$ is partition regular. Let $x=[a,b,c]$
        \[
        \begin{bmatrix} 1 & 1 & -1 \end{bmatrix}
        \begin{bmatrix} a \\ b \\ c  \end{bmatrix} 
        = a+b-c=0.
        \]
        By Schur's theorem, there will exist a solution $x$ that is monochromatic, thus $A$ is PR.
    \end{example}
\end{frame}

% Small lemma
\begin{frame}{Another View to Rado's Theorem}
    \begin{lemma}
        $A$ is PR if and only if $\lambda A$ is PR for all $\lambda \in \Q \setminus \{0\}$.
    \end{lemma}
    \pause
    \begin{proof}
        It is easy to see that assume $A$ is PR, then there will exists a monochromatic solution $x \in \N^n$ such that $Ax=0$. The associative law implies that $(\lambda A)x = \lambda(Ax) = 0$. Thus, $\lambda A$ is PR. The other direction can be proven simply replacing multiplication with division. 
    \end{proof}
    \pause
    \begin{itemize}
        \item $R_r(A) = R_r(\lambda A), \lambda \in \Q \setminus \{0\}$.
    \end{itemize}

\end{frame}

\begin{frame}{Blank}

\end{frame}




%%%Jack's slides
\begin{frame}{Rado numbers $R_3(a(x-y) = bz)$}
 
\begin{center}
\begin{tabular}{c|cccccccc}
	\diagbox{$b$}{$a$} & 1 & 2 & 3 & 4 &5&6&7&8\\
	\hline
	1 & 14 & 14 & 27 & 64 &125 &216 & 343 & 512\\
	2 & 43 & \textit{14} & 31 & \textit{14} &125 &\textit{27} &343&\textit{64} \\
	3 & 94 & 61 & \textit{14} & 73 &125&\textit{14}&343&512\\
	4 & 173 & \textit{43} & 109 & \textit{14} &141&\textit{31}&343&\textit{14} \\
	5 & 286 & 181 & 186 & 180 &\textit{14} &241 & 343 & 512\\
	6 & 439 & \textit{94} & \textit{43}  & \textit{61} &300&\textit{14}& 379 &\textit{73}\\
	7 & 638 & 428 & 442 & 456 &470&462 &\textit{14}&561\\
	8 & 889 & \textit{173} & 633 & \textit{43}&665&\textit{109}& 644 &\textit{14}\\
	9 & 1198 & 856 & \textit{94}& 892 & 910 & \textit{61} & 896 & 896\\
	10 & 1571 & \textit{286} & 1171 & \textit{181} &\textit{43}& \textit{186} & 1190 &\textit{180}
\end{tabular}
\end{center}
Italicized numbers denote equations that are multiples of an equation whose Rado number is already computed.
\end{frame}

\begin{frame}{Conjecture 1}
\begin{center}
	\begin{tabular}{c|cccccccc}
		\diagbox{$b$}{$a$} & 1 & 2 & 3 & 4 &5&6&7&8\\
		\hline
		1 & {\color{red}{14}} & 14 & 27 & 64 &125 &216 & 343 & 512\\
		2 & {\color{red}{43}} & \textit{14} & 31 & \textit{14} &125 &\textit{27} &343&\textit{64} \\
		3 & {\color{red}{94}} & 61 & \textit{14} & 73 &125&\textit{14}&343&512\\
		4 & {\color{red}{173}} & \textit{43} & 109 & \textit{14} &141&\textit{31}&343&\textit{14} \\
		5 & {\color{red}{286}} & 181 & 186 & 180 &\textit{14} &241 & 343 & 512\\
		6 & {\color{red}{439}} & \textit{94} & \textit{43}  & \textit{61} &300&\textit{14}& 379 &\textit{73}\\
		7 & {\color{red}{638}} & 428 & 442 & 456 &470&462 &\textit{14}&561\\
		8 & {\color{red}{889}} & \textit{173} & 633 & \textit{43}&665&\textit{109}& 644 &\textit{14}\\
		9 & {\color{red}{1198}} & 856 & \textit{94}& 892 & 910 & \textit{61} & 896 & 896\\
		10 & {\color{red}{1571}} & \textit{286} & 1171 & \textit{181} &\textit{43}& \textit{186} & 1190 &\textit{180}
	\end{tabular}
\end{center}
\begin{conjecture}[Myers '15]
	$R_3(x-y = bz) = (b+2)^3 - (b+2)^2 - (b+2) - 1$
\end{conjecture}
\end{frame}

\begin{frame}{Generalized Schur numbers}
\begin{definition}
	The \emph{generalized Schur number} $S(k,m)$ is the Rado number $$R_k(x_1+x_2+\dots+x_{m-1} = x_m).$$ 
	\end{definition}
\begin{conjecture}
	$S(k,m) = m^3-m^2-m-1.$
\end{conjecture}
\pause 
\begin{itemize}
\item The equation $x-y = (m-2)z$ is equivalent to $x_1 + (m-2)x_{m-1} = x_m$
\pause \item Setting $x_2 = x_3 = \dots = x_{m-1}$ shows that $S(3,m) \le R_k(x_1 + (m-2)x_{m-1} = x_m)$
\pause \item Myers's conjecture implies the conjecture for generalized Schur numbers 
\end{itemize}

\end{frame}
\begin{frame}{Conjecture 2}
\begin{center}
\begin{tabular}{c|cccccccc}
	\diagbox{$b$}{$a$} & 1 & 2 & 3 & 4 &5&6&7&8\\
	\hline
	1 & 14 & 14 & {\color{red}27} & {\color{red}64} &{\color{red}125} &{\color{red}216} & {\color{red}343} & {\color{red}512}\\
	2 & 43 & \textit{14} & 31 & \textit{14} &{\color{red}125} &\textit{27} &{\color{red}343}&\textit{64} \\
	3 & 94 & 61 & \textit{14} & 73 &{\color{red}125}&\textit{14}&{\color{red}343}&{\color{red}512}\\
	4 & 173 & \textit{43} & 109 & \textit{14} &141&\textit{31}&{\color{red}343}&\textit{14} \\
	5 & 286 & 181 & 186 & 180 &\textit{14} &241 & {\color{red}343} & {\color{red}512}\\
	6 & 439 & \textit{94} & \textit{43}  & \textit{61} &300&\textit{14}& 379 &\textit{73}\\
	7 & 638 & 428 & 442 & 456 &470&462 &\textit{14}&561\\
	8 & 889 & \textit{173} & 633 & \textit{43}&665&\textit{109}& 644 &\textit{14}\\
	9 & 1198 & 856 & \textit{94}& 892 & 910 & \textit{61} & 896 & 896\\
	10 & 1571 & \textit{286} & 1171 & \textit{181} &\textit{43}& \textit{186} & 1190 &\textit{180}
\end{tabular} 
\end{center}

\begin{conjecture}
	If $\gcd(a,b) = 1$ and $a\ge b+2$, then $R_3(a(x-y)=bz) = a^3$. 
\end{conjecture}
\end{frame}

\begin{frame}{Conjecture 2}
\begin{theorem}[Landman, Robertson '03]
	$R_2(a(x-y) = bz) = a^2$ if $\gcd(a,b) = 1$ and $a\ge b+1$. 
\end{theorem}

\pause
\begin{theorem}[CW]
	If $\gcd(a,b) = 1$, then $R_k(a(x-y) = bz) \ge a^k$ for all $k$. 
\end{theorem}
\pause
\begin{proof}
	The $a-$adic valuation $\nu_a: [a^k-1] \to \{0,1,\dots,k-1\}$ gives a valid coloring of $[a^k-1]$. 
\end{proof}
\begin{example}
	$a = 3, b = 1$:
	\begin{align*}
	&{\color{red} 1,2,4,5,7,8,10,11,13,14,16,17,19,20,22,23,25,26} \\
&	{\color{blue} 3,6,12,15,21,24} \\
	&{\color{green} 9,18} \\ 
	\end{align*}
\end{example}
\end{frame}

\begin{frame}{Conjecture 2}
\begin{itemize}
\item Evidence of a stronger conjecture: $R_4(4(x-y) = z) = 256 = 4^4$ and $R_4(5(x-y) = z) = 625 = 5^4$. 
\item \pause Proving the upper bound is more challenging. 
\item \pause Not obvious how to generalize the Landman/Robertson proof for $k = 2$ case

\pause \begin{lemma}[CW] 
Any 3-coloring $\chi$ of $[a^3]$ that avoids monochromatic solutions to $a(x-y) = z$ must satisfy $\chi(a) \neq \chi(a^2)$ 
\end{lemma}

\end{itemize}
\end{frame}

\begin{frame}{Conjecture 3}
 
\begin{center}
	\begin{tabular}{c|cccccccc}
		\diagbox{$b$}{$a$} & 1 & 2 & 3 & 4 &5&6&7&8\\
		\hline
		1 & 14 & 14 & 27 & 64 &125 &216 & 343 & 512\\
		2 & 43 & \textit{14} & {\color{red}31} & \textit{14} &125 &\textit{27} &343&\textit{64} \\
		3 & 94 & 61 & \textit{14} & {\color{red}73} &125&\textit{14}&343&512\\
		4 & 173 & \textit{43} & 109 & \textit{14} &{\color{red}141}&\textit{31}&343&\textit{14} \\
		5 & 286 & 181 & 186 & 180 &\textit{14} &{\color{red}241} & 343 & 512\\
		6 & 439 & \textit{94} & \textit{43}  & \textit{61} &300&\textit{14}& {\color{red}379} &\textit{73}\\
		7 & 638 & 428 & 442 & 456 &470&462 &\textit{14}&{\color{red}561}\\
		8 & 889 & \textit{173} & 633 & \textit{43}&665&\textit{109}& 644 &\textit{14}\\
		9 & 1198 & 856 & \textit{94}& 892 & 910 & \textit{61} & 896 & 896\\
		10 & 1571 & \textit{286} & 1171 & \textit{181} &\textit{43}& \textit{186} & 1190 &\textit{180}
	\end{tabular}
\end{center}
\pause \begin{conjecture}
	$R_3(a(x-y) = (a-1)z) = a^3+(a-1)^2$ for $a \ge 3$. 
\end{conjecture}
\end{frame}
\begin{frame}{Conjecture 3}
\begin{theorem}[CW]
	$R_a(a(x-y) = (a-1)z) \ge a^3+(a-1)^2$ for $a\ge 3$.
\end{theorem}
\begin{proof}
	The following coloring function works:
	
	$$\chi(i) = \begin{cases}
	\text{red} & \text{if } \nu_a(i) = 1 \\
	\text{blue} & \text{if } \nu_a(i) = 2 \text{ or } (\nu_a(i) = 0 \text{ and } (i < a^2 - a \text{ or } i>a^3-a))\\
	\text{green} & \text{otherwise.} 
	\end{cases}$$
\end{proof}
\end{frame}
\begin{frame}{Conjecture 3}
\begin{example}
	The following coloring avoids monochromatic solutions to $3(x-y) = 2z$: 
	\vspace{15pt}
	
\small{	{\color{blue} 1,2},{\color{red}3},{\color{blue} 4,5},{\color{red}6},{\color{green} 7,8},{\color{blue} 9},{\color{green} 10,11},{\color{red}12},{\color{green} 13,14},{\color{red}15},{\color{green} 16,17},{\color{blue} 18},{\color{green} 19,20},{\color{red}21},{\color{green} 22,23},{\color{red}24},{\color{blue} 25,26},{\color{green}27},{\color{blue}28,29},{\color{red}30}}


\end{example}
\end{frame}

\begin{frame}{Regularity of $ax+by+cz = 0$}
\begin{itemize}
\item We know from Rado's theorem that if some nonempty subset of $\{a,b,c\}$ sums to 0, then the equation $\mathcal{E}_{a,b,c} := (ax+by+cz = 0)$ is regular. 

\item \pause But we can still compute the numbers $R_k(\mathcal{E}_{a,b,c})$ if $\mathcal{E}_{a,b,c}$ is $k-$regular. 

\pause 
\begin{definition}The {\color{blue}{degree of regularity} ($dor$)} of $\mathcal{E}_{a,b,c}$ is the largest number $k$ such that $\mathcal{E}_{a,b,c}$ is $k-$regular but not $(k+1)-$regular.
\end{definition}
\end{itemize}

\pause
\begin{problem}
	Determine $dor(\mathcal{E}_{a,b,c})$ for all $a,b,c$. 
\end{problem}
\end{frame}

\begin{frame}{Regularity of $ax+by+cz = 0$}
\begin{theorem}[Rado]
	If $a/b \neq 2^k$, then $dor(a(x+y) = bz) \le 3$. 
\end{theorem}

\begin{itemize}
	\item \pause Computing $R_3(a(x+y) = bz)$ shows that $dor(\mathcal{E}_{a,a,-b}) = 3$. 
\end{itemize}
\pause 
Table of Rado numbers $R_3(a(x+y) = bz)$
\begin{center}
	\begin{tabular}{c|cccc}
		\diagbox{$b$}{$a$} & 1 &2 &3&4 \\
		\hline
		1 & 14 & $>600$ & $>600$ &$>600$ \\
		2& 1  &\textit{14} & 243 &$>$\textit{600}\\
		3& 54 & 54 &\textit{14} & 384 \\
		4& $>600$  & \textit{1} & 108 & \textit{14} \\ 
	\end{tabular}
\end{center}
\end{frame}

\begin{frame}{Goals}
\begin{itemize}
	\item Continue computing $dor(\mathcal{E}_{a,b,c})$
	\item \pause Prove upper bounds for the three conjectures
	\item \pause Do more computation/experimentation with different families of 3-color Rado numbers 
	\item \pause Extend results/experiments to 4 colors 
\end{itemize}
\end{frame}
\end{document}